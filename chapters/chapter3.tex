\chapter{Mathematical Finance}

\section{Market Model}
For the time being consider $r = (r_{t})_{t\in [0,T]}$ to be a deterministic interest rate process. Furthermore assume that we have the following probability space $(\Omega, \F, (\bar{\F}_{t})_{t\in [0,T]}, P)$

\begin{definition}[Money market account]
We define the money market account $B(t)$ as a solution to the ODE: 
$$
dB(t) = r(t)B(t)dt
$$
with initial condition $B(0) = 1$, this gives the solution:
$$
B(t) = e^{\int_{0}^{t}r(s)ds}
$$
\end{definition}

Consider the following processes:
\begin{itemize}
    \item $B = (B_{t})_{t\in [0,T]}$ the money market account
    \item $
    dS_{t} = \mu(t,S_{t})S_{t}dt + \sigma(t,S_{t})S_{t}dW_{t}$ the risky asset. 
    
\end{itemize}

Let $\mu$ and $\sigma$ be defined so that the conditions in Theorem \ref{thm: SDE_sufficiency} are met. 
\\~\\ 
Let $\phi^{i} = \{\phi_{t}^{i}, t \in [0,T]\}$ be two stochastic processes defined on the above probability space. Denote $\phi = (\phi^{0}, \phi^{1})$, where:
\begin{itemize}
    \item $\phi_{t}^{0}$ represents the number of units invested in the money market account at time $t$.
    \item $\phi_{t}^{1}$ represents the number of units invested in the risky asset $S$ at time $t$. 
\end{itemize}

\begin{definition}[\textbf{Trading strategy}]
We say that $\phi = (\phi^{0}, \phi^{1})$ is a trading strategy is it is $(\F_{t})_{t\in [0,T]}$-adapted and: 
\begin{align*}
\phi^{0}rB \in M^{1}([0,T]), \;\; \phi \mu S \in M^{1}([0,T])\;\; \phi^{1}\sigma S \in M^{2}([0,T])    
\end{align*}
\end{definition}

\begin{definition}[\textbf{Value of portfolio}]
The value of a portfolio with trading strategy $\phi$ is given by:
$$
V^{\phi}(t,S_{t}) = \phi_{t}^{0}B_{t} + \phi_{t}^{1}S_{t},\;\; t \in[0,T]
$$
\end{definition}


\begin{definition}[\textbf{Self-financing strategy}]
We say that the trading strategy $\phi$ is self-financing if:
$$
dV^{\phi}(t,S_{t}) = \phi_{t}^{0}dB_{t} + \eta_{t}^{1}dS_{t}, \;\; t\in [0,T] 
$$
\end{definition}

\begin{definition}[\textbf{Arbitrage opportunity}]
An arbitrage opportunity is a self-financing strategy $\phi$ with:
\begin{align*}
V^{\phi}(0,S_{0}) = 0, \;\; V^{\phi}(T,S_{T}) \geq 0, \;\; P(V^{\phi}(T,S_{T}) \geq 0) > 0    
\end{align*}
\end{definition}


\section{Fundamental theorems of asset pricing}

\begin{theorem}[\textbf{First Fundamental theorem of asset pricing}]
\label{thm: First_fundamental_thm_asset_price}
The following are equivalent:
\begin{enumerate}[label=\roman*]
    \item There are no arbitrage opportunities
    \item there exists an equivalent martingale measure $Q\sim P$ such that the process $(\Tilde{S})_{t \in [0,T]} = \left(\frac{S_{t}}{B_{t}}\right)_{t\in [0,T]}$ is a $(Q, \mathbbm{F})$-martingale. 
\end{enumerate}
\end{theorem}

\begin{definition}[\textbf{Attainable claim}]
\label{def: attainable_claim}
We say that a claim $H$ is attainable if there exists a trading strategy $\phi = (\phi^{0}, \phi^{1})$ such that:
$$
V^{\phi}(T,S_{T}) = H \; a.s
$$
We assume that $H$ is $\F_{T}$-measurable as well as $H\in M^{2}([0,T])$
\end{definition}

\begin{definition}[\textbf{Complete market}]
We say the market is complete if all contingent claims in Definition \ref{def: attainable_claim} are attainable.     
\end{definition}

\begin{theorem}[\textbf{Second Fundamental Theorem of Asset Pricing}]
An arbitrage-free market is complete if and only if there exists a unique equivalent martingale measure $Q\sim P$.    
\end{theorem}