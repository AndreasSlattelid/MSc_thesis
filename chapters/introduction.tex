\chapter{Outline}
\label{intro}
\begin{comment}
The rest of the text is organised as follows. This is an example of a description list; see more lists in \cref{discussion} and \vref{tab14}.    
\end{comment}

The thesis is organized as follows: 
\begin{description}
    \item[\cref{chp_theoretical_background}] Consists of the necessary theoretical background for the thesis. This includes measure theory, Probability theory and stochastic analysis. 

    \item[\cref{chp_mathematical_finance}] Introduces important concepts from mathematical finance, like the fundamental theorems of asset pricing. 

    \item[\cref{chp_interest_rate_theory}] Consists of interest rate theory. Here we introduce the zero coupon bond, interest rate swaps, short rate models, HJM framework, and the outgoing LIBOR rates.

    \item[\cref{chp_SOFR}] We establish what SOFR is, including definitions, the difference between 1-month and 3-month SOFR futures, Black and Scholes Option methodology, Swaps and some specific hedges.

    \item[\cref{chp_ESG_swaps}] Here, we introduce an approach for incorporating ESG into Interest-rate swaps. We take one particular case study from real life and use it as motivation for establishing a mathematical framework. 

    \item[\cref{chp_Num_sim}] To better grasp how ESG-swaps could work in general, we have included a numerical simulation. Here, we test different scenarios and see how the ESG framework responds.     

    
\begin{comment}
    \item[\cref{intro}] consists of an interesting introduction. Zooming in on your research question, taking history into account and clarify what your problem is. You could consider calling this a part if you have multiple result parts in your thesis. It ends with an outline.
    \item[\cref{background}] is all about the theoretical background, methods and notation. Don't forget a description of data. 
    \item[\cref{results}] is all about your results. It could be theorems and proofs, it could be applications, it could be examples.
    \item[\cref{discussion}] is more about your results; in subjects that follow the IMRaD-structure, it is known as \enquote{discussion}. Put your results in context. Expand, explain and compare. Weaknesses and strengths of your results. Synergy effects from your results?  Perhaps this chapter is not not relevant for you.
    \item[\cref{conc}] is your fabulous conclusion. Summing up what you have done. Focus on your contribution and repeat how your results make the world a better place. Point to future reseach.
    \item At the end of the thesis, you include short appendices with code, pseudocode, large tables, more examples or figures. See \cref{sec:first-app} and \cref{sec:second-app}. If the length of your appendices is disproportional to the length of your text, consider making it available online with a permanent DOI (Digital Object Identifier), e.g., online repositories like GitHub, Zenodo etc. This will help your future readers and your research group immensely.
\end{comment}
\end{description}
