\chapter{Scripts Chapter 5}
%New colors defined below
\definecolor{codegreen}{rgb}{0,0.6,0}
\definecolor{codegray}{rgb}{0.5,0.5,0.5}
\definecolor{codepurple}{rgb}{0.58,0,0.82}
\definecolor{backcolour}{rgb}{0.95,0.95,0.92}

%Code listing style named "mystyle"
\lstdefinestyle{mystyle}{
  backgroundcolor=\color{backcolour}, commentstyle=\color{codegreen},
  keywordstyle=\color{magenta},
  numberstyle=\tiny\color{codegray},
  stringstyle=\color{codepurple},
  basicstyle=\ttfamily\footnotesize,
  breakatwhitespace=false,         
  breaklines=true,                 
  captionpos=b,                    
  keepspaces=true,                 
  numbers=left,                    
  numbersep=5pt,                  
  showspaces=false,                
  showstringspaces=false,
  showtabs=false,                  
  tabsize=2
}

%\section{R-code to generate ESG-fixed rate path}
%\lstset{style=mystyle}
%\lstinputlisting[language = R, caption = {Estimating Term Structure}]{scripts/R/SBM_ESG_process.R}


%\section{R-code to generate Term Structure}
%\lstset{style=mystyle}
%\lstinputlisting[language = R, caption = {Estimating Term Structure}]{scripts/R/Nelson_Siegel.R}


%\newpage 

%\section{R-code to generate ESG-figures}
%\lstset{style=mystyle}
%\lstinputlisting[language = R, caption = {Generate ESG-risk score}]{scripts/R/criteria_plt.R}

\section{SOFR: Dynamics of $f^{1M}(t,S_{1M}, T_{1M})$ and $f^{3M}(t,S_{3M}, T_{3M})$}

\begin{minted}{julia}
using Revise

#stats etc.
using Random 
using Distributions

#plotting histogram and LaTeX labels:
using Plots 
using LaTeXStrings

#Vasicek dynamics of SOFR futures rates:
alpha = 0.30
sigma = 0.03

function B(t,S,T)
    return (1/alpha)*(exp(-alpha*(S-t))-exp(-alpha*(T-t)))
end


function f1M(t,S,T, dt, initial)
    "
    Simulates 1M-futures rates for t<= S
    Args:
        t {Float64}: initial start point 
        S {Float64}: start observation period 
        T {Float65}: end observation period 
        dt {Float64}: stepsize 
        inital {Float64}: inital futures rate
    Returns: 
        df1M(t,S,T) = 1/(T-S)*B(t,S,T)*sigma*dW^{Q}(t)
    "
    time = range(t, S, step=dt)
    n = length(time)

    Random.seed!(1)
    Z = rand(Normal(0,1), n)
    f1_r = zeros(n)
    f1_r[1] = initial
    for i in 2:n
        f1_r[i] = f1_r[i-1] + (1/(T-S))*B(time[i-1], S,T)*sigma*sqrt(dt)*Z[i]
    end

    return f1_r
end


function f3M(t,S,T, dt, initial)
    "
    Simulates 3M-futures rates for t<= S
    Args:
        t {Float64}: initial start point 
        S {Float64}: start observation period 
        T {Float65}: end observation period 
        dt {Float64}: stepsize 
        inital {Float64}: inital futures rate
    Returns: 
        df3M(t,S,T) = (f3M(t,S,T)+1/(T-S))*B(t,S,T)*sigma*dW^{Q}(t)
    "
    time = range(t, S, step=dt)
    n = length(time)

    Random.seed!(2)
    #W(t) d= sqrt(t)Z, Z ~ N(0,1)
    Z = rand(Normal(0,1), n)
    f3_r = zeros(n)
    f3_r[1] = initial
    for i in 2:n
        f3_r[i] = f3_r[i-1] + 
        (f3_r[i-1] + 1/(T-S))*B(time[i-1], S,T)*sigma*sqrt(dt)*Z[i]
    end

    return f3_r
end

Random.seed!(3)
#time params: 
t = 0
dt = 1/360
#1M-futures:
S1M = 6/12
T1M = S1M + 1/12
#3M-futures: 
S3M = S1M
T3M = S3M + 3/12


#simulation 
time = range(t, S1M, step=dt)
n = length(time)
Z = rand(Normal(0,1), n)
f1 = zeros(n)
f3 = zeros(n)


f1[1] = (100-95.025)*1/100
f3[1] = (100-95.16)*1/100

for i in 2:n
    f1[i] = f1[i-1] + (1/(T1M-S1M))*B(time[i-1], S1M,T1M)*sigma*sqrt(dt)*Z[i]
    f3[i] = f3[i-1] + (f3[i-1] + 1/(T3M-S3M))*B(time[i-1], S3M,T3M)*sigma*sqrt(dt)*Z[i]
end

plot(f1, label = L"f^{1M}(t, S_{1M},T_{1M})", title =
L"\alpha = 0.30,\; \sigma = 0.03,\; t\in [0,S_{1M}]", legend= :topleft)
plot!(f3, label = L"f^{3M}(t, S_{3M}, T_{3M})")
xticks!([0, n/2 ,n], ["0", L"\frac{S_{1M}}{2}", L"S_{1M}"])

\end{minted}


\newpage 
\section{SOFR: Simulation of $\kappa_{t}^{3M-SOFR}$}
\begin{minted}{julia}
using Revise

#statistics and distributions
using Random 
using Distributions
using Statistics

#data-wrangeling:
using DataFrames

#for numerical integration:
using QuadGK

#plotting histogram and LaTeX labels:
using Plots 
using LaTeXStrings

#---------------------------------------------------------------------
#time parameters
T0 = 1/12
T1 = 4/12
T2 = 7/12
T3 = 10/12
timepoints = [T0, T1, T2, T3]
n_steps = 10000

#We use that Vasicek is ATS, i.e P(t,T) = exp(-A(t,T)-B(t,T)r(t))

function r_Vasicek(alpha, m,sigma, r_t, time_interval, n_steps)
    """
    Args: 
        #Vasicek parameters:
        alpha{Float64}: speed of reversion 
        m{Float64}: long term mean level 
        sigma{Float64}: volatility 
        r_t{Float64}: initial value of r = (r(u))
        
        #time:
        time_interval (vector): the time interval we model measured in years.
        n_steps (int): number of timesteps we partition over
    
    Returns: 
        it simulates the process: r = (r(u)), for u in [t_start, t_end]
    """

    #time partition: 
    t_start = time_interval[1]
    t_end = time_interval[2]
    dt = (t_end-t_start)/n_steps
    n_steps = length(collect(t_start:dt:t_end))

    #initializing r: 
    r = zeros(n_steps)
    r[1] = r_t

    #Standard normal rv's
    Z = rand(Normal(0,1), n_steps)

    for i in 2:n_steps
        r[i] = r[i-1] - alpha*(m-r[i-1])*dt + sigma*sqrt(dt)*Z[i]
    end

    return r
end

function B_ZCB(t,T)
    ans = -(1/alpha)*(exp(-alpha*(T-t))-1)
    return ans
end

function A_ZCB(t,T)
    integral, _ = quadgk(u -> B_ZCB(u,T)^(2), t,T)
    ans = m*B_ZCB(t,T) - m*(T-t) - (1/2)*sigma^(2)*integral 
    return ans
end

function P(t,T, r_t)
    ans = exp(-A_ZCB(t,T) -B_ZCB(t,T)*r_t)
    return ans
end

#--------------------------------------------------------------
#Calculating f^3M(t,S,T), again using ATS structure:
#f^3M(t,S,T) = 1/(T-S)*(exp(A(t,S,T) + B(t,S,T)r(t))-1)
function Sigma1(u,t,S,T)
    ans = exp(-alpha*(S-u)) - exp(-alpha*(T-u))
    return ans   
end

function Sigma2(u,t,S,T)
    ans = 1-exp(-alpha*(T-u))
    return ans
end

function B(t,S,T)
    ans = (1/alpha)*(exp(-alpha*(S-t))-exp(-alpha*(T-t)))
    return ans
end

function A(t,S,T)
    first_part = m*(T-S) - m*B(t,S,T)
    c1_2, _  = quadgk(u -> Sigma1(u, t, t, S)^(2), t,S) 
    c2_2, _  = quadgk(u -> Sigma2(u, t, S, T)^(2), S,T)

    ans = first_part + (1/2)*(sigma^2/alpha^2)*(c1_2 + c2_2)

    return ans
end


function f_3M(t,S,T, r_t)
    ans = (1/(T-S))*(exp(A(t,S,T) + B(t,S,T)*r_t) - 1)
    return ans
end 

#time t-value of kappa in 3M SOFR-futures rate swap
function kappa_t(t, r_t)
    "
    Args: 
        t{Float64}: vector of time points i.e [0,T1]
        r_t{Float64}: vector of realization of interest rate model
    Returns: 
        above = sum(P(t,T_{i}*f^{3M}(t,T_{i-1}, T_{i})), i = 1:n)
        below = sum(P(t,T_{i}), i = 1:3)
        kappa_t_3M_SOFR = above/below
    "
    ZCB_prices = map(T -> P(t,T, r_t), timepoints[2:end])
    f_3M_rates = map((x, y) -> f_3M(t, x, y, r_t), timepoints[1:end-1], timepoints[2:end])
    above = sum(ZCB_prices.*f_3M_rates)
    below = sum(ZCB_prices) 
    ans = above/below
    return ans
end


#Vasicek parameters: 
alpha = 0.25
m = 0.035
r_0 = 0.0425
sigma = 0.02

#time:
t_start = 0 
t_end = T0
dt = (t_end-t_start)/n_steps
t = collect(t_start:dt:t_end)

#initialization of simulation
n_sim = 2

R = zeros(length(t), n_sim) #Vasicek rates
K = zeros(length(t), n_sim) #fixed rate kappa for each pair (t,r_t)

Random.seed!(1234)
for i in 1:n_sim
    #Vasicek realization:
    r = r_Vasicek(alpha, m, sigma, r_0, [t_start,t_end], n_steps)
    #collect the time t rate and time t kappa:
    R[:, i] = r
    K[:, i] = map((x,y)-> kappa_t(x,y), t, r)
end

R
K[1]

#plot of rates
plot(R, layout = (1,1), 
        legend = false,
        title = L"t \mapsto r(t),\alpha = 0.25, m = 0.035, \sigma = 0.02, r_{0} = 0.0425 "
        )
xticks!([0, 10_000/2 ,10_000], ["0", L"\frac{T_{0}}{2}", L"T_{0}"])


#plot of kappa_t
plot(K, layout=(1,1), 
        legend = false, 
        title = L"t \mapsto \kappa_{t}^{3M-SOFR},\alpha = 0.25, m = 0.035, \sigma = 0.02, r_{0} = 0.0425"
        )
xticks!([0, 10_000/2 ,10_000], ["0", L"\frac{T_{0}}{2}", L"T_{0}"]) 



\end{minted}

\newpage 

\section{SOFR: Hedging 3M-arithmetic SOFR}
\begin{minted}{julia}
using Random 
using Distributions
using Statistics
using StatsPlots #qqplot

#for matrix operations and linear programming
using LinearAlgebra
using JuMP  #lp-problem setup
using HiGHS #lp-solver

#data-wrangeling:
using DataFrames

#for numerical integration:
using QuadGK

#rerun calculations easier:
using Revise

#plotting histogram and LaTeX labels:
using Plots 
using LaTeXStrings

#--------------------------------------------------------------------------#
#Vasicek parameters: 
alpha = 0.25
m = 0.035
r_t = 0.0425
sigma = 0.02

#time parameters
t = 0
S = 1/12
T1M = 2/12
T2M = 3/12
T = 4/12

function Sigma1(u,t,S,T)
    ans = exp(-alpha*(S-u)) - exp(-alpha*(T-u))
    return ans   
end

function Sigma2(u,t,S,T)
    ans = 1-exp(-alpha*(T-u))
    return ans
end

function int_r_start_stop(low,up, t)
    "
    The integral: integral(r(u)du, low, up) as described in Eq (5.5) p.66
    Args: 
        low{Float64}: lower integration limit
        up{Float64}:  upper integtation limit
    Returns: 
        the integral: int_low_up r(u)du
    "
    if t > low
        return "Please chose t <=low"
    end

    #Sigma1 is N(0, int_t_low c1_2 du), Sigma2 is N(0, int_low_up c2_2 du)
    c1_2, _  = quadgk(u -> Sigma1(u, t, low,up)^(2), t,low) 
    c2_2, _  = quadgk(u -> Sigma2(u, t, low,up)^(2), low,up)
    
    c1 = sqrt(c1_2)
    c2 = sqrt(c2_2)
    Z = rand(Normal(0,1))
    ans = ((r_t-m)/alpha)*(exp(-alpha*(low-t))-exp(-alpha*(up-t))) + 
                        m*(up-low) + sigma/alpha*(c1*Z + c2*Z)
    return ans
end

function integrand_E_Q_r(u, r_t, t)
    ans = exp(-alpha*(u-t))*r_t + m*(1-exp(-alpha*(u-t)))
    return ans
end

#int_S_T E_Q[r(u)|F_t]du:
integral_E_Q_r, _ = quadgk(u -> integrand_E_Q_r(u, r_t, t), S,T)

integral_E_Q_r
#--------------------------------------------------------------------------#
# Calculating a_hat_3M:
function B(t,S,T)
    ans = (1/alpha)*(exp(-alpha*(S-t))-exp(-alpha*(T-t)))
    return ans
end

function A(t,S,T)
    first_part = m*(T-S) - m*B(t,S,T)
    c1_2, _  = quadgk(u -> Sigma1(u, t, t, S)^(2), t,S) 
    c2_2, _  = quadgk(u -> Sigma2(u, t, S, T)^(2), S,T)

    ans = first_part + (1/2)*(sigma^2/alpha^2)*(c1_2 + c2_2)

    return ans
end


function f_3M(t,S,T)
    " 
    Vasicek representation of f^{3M}(t,S,T) as described in Eq. (5.4) p.60
    "
    ans = (1/(T-S))*(exp(A(t,S,T) + B(t,S,T)*r_t) - 1)
    return ans
end

f_3M(0,S,T)

a_hat = integral_E_Q_r/((T-S)*f_3M(0,S,T))

#-------------------------------------------------------------------------------------------#
# 3M-arithmetic vs (a,b,c) 1M-SOFR futures: 
integral1 , _ = quadgk(u -> integrand_E_Q_r(u, r_t, t), S,T1M)
integral2 , _ = quadgk(u -> integrand_E_Q_r(u, r_t, t), T1M,T2M)
integral3 , _ = quadgk(u -> integrand_E_Q_r(u, r_t, t), T2M,T)

f_1M_S_T1M = (1/(T1M-S))*integral1
f_1M_T1M_T2M = (1/(T2M-T1M))*integral2
f_1M_T2M_T = (1/(T-T2M))*integral3

#variable naming to be more consistent with MSc Thesis:
a = f_1M_S_T1M        #alpha, I use alpha in Vasicek, hence a: 
beta = f_1M_T1M_T2M   #beta
gamma = f_1M_T2M_T    #gamma

futures = [a,beta,gamma]

#E_Q[X^(3M_A)(S,T)|F_t] = q: 
q = (1/(T-S))*integral_E_Q_r 

#matrix of coeff: 
M = [a^(2)      a*beta  a*gamma;
     beta^(2)   a*beta  beta*gamma; 
     gamma^(2)  a*beta  beta*gamma] 
     
#vector of values: 
b = q.*[a;
        beta;
        gamma] 

#optimal weight of futures:
x_hat = inv(M)*b 
#-------------------------------------------------------------------------------------------#
# incase M is not invertible: 
# Define optimization problem
model = Model(HiGHS.Optimizer)
@variable(model, x[1:3])
@objective(model, Min, sum(x))
@constraint(model, M * x .== b)

# Solve optimization problem
optimize!(model)
# optimal value
x_tilde = value.(x)

#-------------------------------------------------------------------------------------------#
# Simulations: 
n_sim = 10^(6)
#constants: 
#int_S_T E_Q[r(u)|F_t]du:
integral_E_Q_r, _ = quadgk(u -> integrand_E_Q_r(u, r_t, t), S,T)

futures_weighted_M_inv = x_hat'futures
futures_weighted_BP = x_tilde'futures


Random.seed!(1234)
X_3MA = zeros(n_sim)
for i in 1:n_sim
    #aritmetic interest rate relaization:
    X_3MA[i] = (1/(T-S))*(int_r_start_stop(S,T,t))
end

mean(X_3MA)
#elementwise substraction:
ER_1 = X_3MA .-(1/(T-S))*integral_E_Q_r
ER_2_M_inv = X_3MA .-futures_weighted_M_inv
ER_2_random = X_3MA .-[0.33, -0.33, 0.33]'futures

#-----------------------------------------------------------
# plotting of histograms: 
#hedge with a_{t}^{3M}- f^{3M}
mean_ER_1 = round(mean(ER_1), digits = 3)
sigma_ER_1 = round(std(ER_1), digits = 2) 

#hedge with optimal (a_{t}^{1M}, b_{t}^{1M}, c_{t}^{1M})-f^{1M}
mean_ER_2_M_inv = round(mean(ER_2_M_inv), digits = 3)
sigma_ER_2 = round(std(ER_2_M_inv), digits = 2)

#not optimal 1M hedges, naive strategy:  
mean_ER_2_random = round(mean(ER_2_random), digits = 3)
sigma_ER_2_random = round(std(ER_2_random), digits = 2)

#ER_1
ticks_ER_1 = round.([mean_ER_1 + i*sigma_ER_1 for i in -3:1:3], digits = 3)

histogram(ER_1, 
          color =:lightblue, 
          xlabel="Value", 
          ylabel="Frequency", 
          title = 
          L"Histogram\; of\; ER_{1}(0), \; s_{ER_{1}}\approx 0.01,\;n_{sim} = 10^{6}", 
          labels = "ER_1(0)", 
          xticks = ticks_ER_1
          )
vline!([mean_ER_1], lw = 5, labels = L"mean(ER_{1}(0))" )


#ER_2_M_inv:
ticks_ER_2_M_inv = round.([mean_ER_2_M_inv + i*sigma_ER_2 for i in -3:1:3], digits = 3)

histogram(ER_2_M_inv, 
          color =:lightblue, 
          xlabel="Value", 
          ylabel="Frequency", 
          title = 
          L"Histogram\; of\; ER_{2}^{M_{inv}}(0), \; s_{ER_{2}^{M_{inv}}} \approx 0.01, \;n_{sim} = 10^{6}", 
          labels = L"ER_{2}^{M_{inv}}(0)", 
          xticks = ticks_ER_2_M_inv
          )
vline!([mean_ER_2_M_inv], lw = 5, labels = L"mean(ER_{2}^{M_{inv}}(0))")


#Naive strategy ER_2_random (0.33, -0.33, 0.33)
ticks_random = round.([mean_ER_2_random + i*sigma_ER_2_random for i in -2:1:2], digits = 3)

histogram(ER_2_random, 
          color =:lightblue, 
          xlabel="Value", 
          ylabel="Frequency", 
          title = 
          L"Hist\; of\; ER_{2}^{(\hat{a}_{0}, \hat{b}_{0}, \hat{c}_{0})}(0), \; s_{ER_{2}^{(\hat{a}_{0}, \hat{b}_{0}, \hat{c}_{0})}} \approx 0.01, \;n_{sim} = 10^{6}", 
          labels = L"ER_{2}^{(0.33,-0.33,0.33)}(0)", 
          xticks = ticks_random
          )
vline!([mean_ER_2_random], lw = 5, labels =  L"mean(ER_{2}^{(0,0,1)}(0))")



#adressing normality:
#-----------------------------------------
x = ER_1[1:10^(6)]
y = rand(Normal(mean_ER_1, sigma_ER_1), 10^(6))

qqplot(x,y, title =  
            L"(Q-Q)\; plot\; of\; ER_{1}(0) \;vs\; \mathcal{N}\left(\overline{ER_1(0)}, s_{ER_1(0)}^{2})\right)", 
            xlabel = "Theoretical Quantiles", 
            ylabel = "Sample Quantiles")
\end{minted}



%\lstset{style=mystyle}
%\lstinputlisting[language = julia, caption = {ESG-swap rate Simulation}]{ZCB_ESG_for_latex.jl}

%\jlinputlisting{ZCB_ESG_for_latex.jl}

%\lstinputlisting[language= Julia]{ZCB_ESG_for_latex.jl}


% displayed code
%\begin{jllisting}
%# some julia code
%println( "Here we go with Julia!")
%\end{jllisting}
