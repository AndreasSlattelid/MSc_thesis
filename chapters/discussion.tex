\chapter{Discussion}

We imposed a model for ESG-linked swaps, which lead to a sequence of fixed rates $\kappa_{t}^{ESG}  = (\kappa_{t}^{ESG}(i))_{i=1}^{n}$ giving a penalty/discount depending on whether the criteria were met or not.
\\~\\ 
We remember that the criteria $A_{i}$ looked like the following:
\begin{align*}
A_{i} &= \{X_{T_{i}} \leq C_{T_{i}}^{ESG}\}    
\end{align*}

This means that our ESG-fixed rate process heavily depends upon the OU-process $X(t)$ and the criteria $C_{T_{i}}^{ESG}$. 
\\~\\
In my modelling approach, I modelled directly under $Q$, however, the market one operates in and observes is under $P$, meaning that it would have been suitable with an Esscher-transform of $X(t)$, which then by  Proposition \ref{prop: Esscher_transform_CPP_Q} means that we would still have a CPP, but with altered intensity $\lambda_{Q}$ and jump-size distribution $F_{J}^{Q}(dx)$.
\\~\\
In our case we have $I(t) = \sum_{k=1}^{N(t)}J_{k}$ with $N(t) \sim Pois(\lambda t)$ and 
$J\sim Exp(\mu)$, meaning that we would have gotten: 

\begin{align*}
Mathematics
\end{align*}

Furthermore, $X(t)$ is company dependent, meaning that in order to get good estimates of the necessary parameters included, the need for data accessibility is crucial in order to impose a suitable model. In the Julia script it is possible to extend $X(t)$ with a Brownian Motion, i.e: 

\begin{align*}
d\hat{Z}(t) &= -\beta \hat{Z}(t)dt + \sigma dW(t) + dI(t)\\ 
\hat{X}(t) &= \exp\left(\hat{Z}(t)\right)
\end{align*}


\newpage 
If one looks at the expression $K_{n}^{ESG}(\omega)$: 
\begin{align*}
K_{n}^{ESG}(\omega) &= 
[\kappa -dn]\mathbbm{1}\left[
\bigcap_{i\in \mathcal{I}_{n}}A_{i}
\right](\omega) \\ 
&+ 
\sum_{\alpha \in \mathcal{I}_{2n}^{Even}}
\left(
[\kappa -d(n-\alpha)]\mathbbm{1}\left[
\bigcup_{
j_{1}\neq \dots \neq j_{|\mathcal{I}_{\alpha}^{Even}|}
\in \mathcal{I}_{n}
}\left(
\bigcap_{i\in \mathcal{I}_{n}}A_{i}
\right)^{
\{
(j_{1}, \dots , j_{|\mathcal{I}_{\alpha}^{Even}|})
\}
}
\right]
\right)(\omega) 
\end{align*}

We see that it tracks every path, and for each $n$ there are $2^{n}$-possible paths, meaning that as $n$ increases the complexity increases. Furthermore, this expression is rather general meaning that one must rely upon Monte Carlo simulations in order to get an estimate of $\kappa_{t}^{ESG}(i)$. At the same time, it also gives possibilities for other types of stochastic models for the ESG-risk score. 




%\label{discussion}
%\kant[15] % Dummy text
%\section{All my great results}
%\kant[16]
%\paragraph{List with bullets}
%\begin{itemize}
%    \item This is a list with bullets - the symbol can be %changed easily.
%     \item[!] A point to exclaim something!
%  \item[$\blacksquare$] Make the point fair and square.
%  \item[] A blank label?
%    \item This is the last item.
%\end{itemize}
%\paragraph{Numbered lists}
%\begin{enumerate}
%    \item This is a numbered list.
%     \item This is the second item.
%\end{enumerate}
%\paragraph{Descriptions}
%\begin{description}
%\item[My first great result] This is a description list.
% \item[My second great result] This is the second item.
%\end{description}
%\section{Pros and cons}
%\kant[17]
%\section{Future research}
%\kant[18]