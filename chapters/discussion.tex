\chapter{Discussion}
In the SOFR studies, we examined the implications of different underlying calculation methods for the SOFR futures. We studied a 3M-arithmetic product: 
\[
X^{3M_{A}}(S,T) = \frac{1}{T-S}\int_{S}^{T}r_{u}du
\]
Against taking  a $\hat{a}_{t}-f^{3M}(t,S,T)$ 3M-SOFR futures position vs \\ 
taking  a $(\hat{a}_{t}, \hat{b}_{t}, \hat{c}_{t})-f^{1M}(t,S,T)$ 1M-SOFR futures position.  
\\~\\
For simulation purposes, we chose the following model for the integral of $r = (r(t))_{t\geq 0}$: 
\begin{align*}
\int_{S}^{T}r(u)du 
&= 
\left(
\frac{r(t)-m}{\alpha}
\right)
\left[
e^{-\alpha(S-t)} - e^{-\alpha(T-t)}
\right]
+ m(T-S) 
+ 
\frac{\sigma}{\alpha}\int_{t}^{T}\Sigma(u,t,S,T)dW^{Q}(u)   
\end{align*}
Where: 
\begin{align*}
\Sigma(u,t,S,T) &= 
\left[
e^{-\alpha(S-u)}-e^{-\alpha(T-u)}
\right]\mathbbm{1}_{[t,S)}(u) 
+ 
\left[
1-e^{-\alpha(T-u)}
\right]\mathbbm{1}_{[S,T]}(u)
\end{align*}

It should be mentioned that this may not be a realistic representation of $r$, and should be addressed accordingly. For instance, if one looks at the graph in Figure \ref{fig: Overnight_SOFR_rates}.  
\\~\\
From simulations we got $\hat{a}_{t}^{3M} = 0.95$ for the position in 3M-SOFR futures. An ideal hedge here would be if this number equalled $1$. Pretend that instead of hedging $X^{3M_{A}}(S,T)$, we wanted to hedge: 
\begin{align*}
X^{3M_{G}}(S,T) := \frac{1}{T-S}\left[
e^{\int_{S}^{T}r(u)du}-1
\right]    
\end{align*}

Namely a geometric average over the period $[S,T]$, then: 
\begin{align*}
G(a_{t}) := \argmin\limits_{a_{t} \in \R}\E_{Q}\left[
\left(
X^{3M_{G}}(S,T)-a_{t}f^{3M}(t,S,T)
\right)^{2}
\bigg{|}\F_{t}
\right]
\end{align*}

Now following the same arguments as on p.\pageref{result: optimal_SOFR_hedge_3MA_vs_3GM}, we get: 
\begin{align*}
\hat{a}_{t}^{3M} &= \frac{
\E_{Q}[X^{3M_{G}}(S,T)|\F_{t}]
}{
f^{3M}(t,S,T)
}
= 
\frac{
f^{3M}(t,S,T)
}{
f^{3M}(t,S,T)
} = 1
\end{align*}


\newpage 
We imposed a model for ESG-linked swaps, which lead to a sequence of fixed rates $\kappa_{t}^{ESG}  = (\kappa_{t}^{ESG}(i))_{i=1}^{n}$ giving a penalty/discount depending on whether the criteria were met or not.
\\~\\ 
We remember that the criteria $A_{i}$ looked like the following:
\begin{align*}
A_{i} &= \{X_{T_{i}} \leq C_{T_{i}}^{ESG}\}    
\end{align*}

This means that our ESG-fixed rate process heavily depends upon the OU-process $X(t)$ and the criteria $C_{T_{i}}^{ESG}$. It could be hard to establish "reasonable" criteria. In our simulation, we took $C^{ESG} = (C_{T_{i}}^{ESG})_{i\geq 1}$ to be $\F_{0}$-measurable, this could lead to some uncertainty as one would have to "know" even more about the company's development. Maybe a more reasonable approach would be to take $C^{ESG}$ to be $\F_{T_{i-1}}$-measurable. However, this would require a model for $C^{ESG}$, which adds to the complexity again.    
\\~\\
In our modelling approach, we modelled directly under $Q$. However, the market one operates in is under $P$, meaning that it would have been suitable with an Esscher-transform of $X(t)$, which then by  Proposition \ref{prop: Esscher_transform_CPP_Q} means that we would still have a CPP, but with altered intensity $\lambda_{Q}$ and jump-size distribution $F_{J}^{Q}(dx)$.
\\~\\
In our case we have $I(t) = \sum_{k=1}^{N(t)}J_{k}$ with $N(t) \sim Pois(\lambda t)$ and 
$J\sim Exp(\mu)$, and from Lemma \ref{lemma: CPP_exp_mu}, with $\theta \in (-\infty, \mu)$, we have:
\begin{align*}
\lambda_{Q} &= \frac{\lambda \mu}{\theta - \mu} \;\text{and}\;
J \stackrel{Q}{\sim} Exp(\mu - \theta)
\end{align*}

Furthermore, $X(t)$ is company dependent, meaning that to get reasonable estimates of the necessary parameters included, data accessibility is crucial to impose a suitable model. 
\\~\\ 
One should also mention that there are many ESG-rating agencies and different metrics for measuring the ESG score/risk score. Furthermore, these metrics are not necessarily standardized, which adds to the complexity of modelling such a score. 
\\~\\
If one looks at the expression $K_{n}^{ESG}(\omega)$: 
\begin{align*}
K_{n}^{ESG}(\omega) &= 
[\kappa -dn]\mathbbm{1}\left[
\bigcap_{i\in \mathcal{I}_{n}}A_{i}
\right](\omega) \\ 
&+ 
\sum_{\alpha \in \mathcal{I}_{2n}^{Even}}
\left(
[\kappa -d(n-\alpha)]\mathbbm{1}\left[
\bigcup_{
j_{1}\neq \dots \neq j_{|\mathcal{I}_{\alpha}^{Even}|}
\in \mathcal{I}_{n}
}\left(
\bigcap_{i\in \mathcal{I}_{n}}A_{i}
\right)^{
\{
(j_{1}, \dots , j_{|\mathcal{I}_{\alpha}^{Even}|})
\}
}
\right]
\right)(\omega) 
\end{align*}

We see that it tracks every path, and for each $n$ there are $2^{n}$-possible paths, meaning that as $n$ increases, the complexity increases. Furthermore, this expression is rather general, meaning that one must rely upon Monte Carlo simulations to get an estimate of $\kappa_{t}^{ESG}(i)$. At the same time, it also gives possibilities for other types of stochastic models for the ESG-risk score. 




%\label{discussion}
%\kant[15] % Dummy text
%\section{All my great results}
%\kant[16]
%\paragraph{List with bullets}
%\begin{itemize}
%    \item This is a list with bullets - the symbol can be %changed easily.
%     \item[!] A point to exclaim something!
%  \item[$\blacksquare$] Make the point fair and square.
%  \item[] A blank label?
%    \item This is the last item.
%\end{itemize}
%\paragraph{Numbered lists}
%\begin{enumerate}
%    \item This is a numbered list.
%     \item This is the second item.
%\end{enumerate}
%\paragraph{Descriptions}
%\begin{description}
%\item[My first great result] This is a description list.
% \item[My second great result] This is the second item.
%\end{description}
%\section{Pros and cons}
%\kant[17]
%\section{Future research}
%\kant[18]